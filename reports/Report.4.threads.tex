\documentclass{report}
\title{Report 4: Threads}
\author{Nguyen Duc Vu}

\begin{document}

\maketitle

\begin{section}


\begin{verbatim}

1. Improvement on the labwork:
- Modify the kernel to make Device CPU process in 2D: adding y to get the location of each pixel in columns
__global__ void grayscale2d(uchar3 *input, uchar3 *output, int width) {
int x = threadIdx.x + blockIdx.x * blockDim.x;
int y = threadIdx.y + blockIdx.y * blockDim.y;
int tid = y * width + x;
output[tid].x = (input[tid].x + input[tid].y + input[tid].z) / 3;
output[tid].z = output[tid].y = output[tid].x;
}

- Modify the labwork4 code part:
	+ define gridSize and blockSize 
    dim3 gridSize = dim3(8, 8);
    dim3 blockSize = dim3(32, 32);

    + call the grayscale2d function to process the pixels
    grayscale2d<<<gridSize, blockSize>>>(devInput, devOutput, inputImage->width);

    + for the rest of the part, keep it remaining as Labwork3

2. Speed up:


vund@ictserver3:~/advancedhpc2017/labwork/build$ ./labwork 3 ../data/eiffel.jpg
USTH ICT Master 2017, Advanced Programming for HPC.
Warming up...
Starting labwork 3
labwork 3 ellapsed 130.1ms

vund@ictserver3:~/advancedhpc2017/labwork/build$ ./labwork 4 ../data/eiffel.jpg
USTH ICT Master 2017, Advanced Programming for HPC.
Warming up...
Starting labwork 4
labwork 4 ellapsed 98.6ms


--> The function labwork4 is a little bit faster then the labwork3

3. Thread: Exercises

Exercise 1:

- 8x8: --> the blocksize would be 64  -> can be able to fit the threads allowed for each block (512)
- 16x16 --> the blocksize = 256 --> still fit the allowed threads --> this is the best configuration as it is closest to the maximum threads/block given
- 32x32 --> blocksize = 1024 --> greater than maximum threads allow for each block --> not work

Exercise 2:

- 256 threads/blk would result in the most number of threads in the SM as with 4 blocks, the device SM will have to take: 256 x 4 = 1024 threads, which is closest to the maximum threads that the device can handle on this case.

Exercise 3: 
- 2048 threads will be in the grid to cover	the vector length (2000)

\end{verbatim}

\end{section}


\end{section}


\end{document}
