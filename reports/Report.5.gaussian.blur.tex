\documentclass{report}
\title{Report 5: Gaussian Blur Convolution}
\author{Nguyen Duc Vu}

\begin{document}

\maketitle

\begin{section}


\begin{verbatim}

1. Explain the implementation of the Gaussian Blur filter
- create the filter as a fixed size 7x7 matrix: this defines the value of target pixel and its surrounding:
	int kernel[] = { 0, 0, 1, 2, 1, 0, 0,
                     0, 3, 13, 22, 13, 3, 0,
                     1, 13, 59, 97, 59, 13, 1,
                     2, 22, 97, 159, 97, 22, 2,
                     1, 13, 59, 97, 59, 13, 1,
                     0, 3, 13, 22, 13, 3, 0,
                     0, 0, 1, 2, 1, 0, 0 };

- find get the pixelCount:
	int pixelCount = inputImage->width * inputImage->height;

- Do the pixel processing/value calculation part:
	+ Apply the fixed 7x7 matrix filter (kernel) into each target pixel

	+ multiple all elements in the filter with the target pixel and its 6 elements surrounding

	+ then adding all of them together

	+ divide the sum to the total number of elements in the matrix to get the average value pixel

int pixelCount = inputImage->width * inputImage->height;
    outputImage = (char*) malloc(pixelCount * sizeof(char) * 3);
    for (int row = 0; row < inputImage->height; row++) {
        for (int col = 0; col < inputImage->width; col++) {
            int sum = 0;
            int c = 0;
            for (int y = -3; y <= 3; y++) {
                for (int x = -3; x <= 3; x++) {
                    int i = col + x;
                    int j = row + y;
                    if (i < 0) continue;
                    if (i >= inputImage->width) continue;
                    if (j < 0) continue;
                    if (j >= inputImage->height) continue;
                    int tid = j * inputImage->width + i;
                    unsigned char gray = (inputImage->buffer[tid * 3] + inputImage->buffer[tid * 3 + 1] + inputImage->buffer[tid * 3 + 2])/3;
                    int coefficient = kernel[(y+3) * 7 + x + 3];
                    sum = sum + gray * coefficient;
                    c += coefficient;
                }
            }
            sum /= c;
            int posOut = row * inputImage->width + col;
            outputImage[posOut * 3] = outputImage[posOut * 3 + 1] = outputImage[posOut * 3 + 2] = sum;
        }


2. Speed up:

vund@ictserver3:~/advancedhpc2017/labwork/build$ ./labwork 5 ../data/eiffel.jpg
USTH ICT Master 2017, Advanced Programming for HPC.
Warming up...
Starting labwork 5
labwork 5 ellapsed 1780.8ms


\end{verbatim}

\end{section}


\end{section}


\end{document}
