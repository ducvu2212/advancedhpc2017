\documentclass{report}
\title{Report 6: Map}
\author{Nguyen Duc Vu}

\begin{document}

\maketitle

\begin{section}


\begin{verbatim}

1. Implementation
- I reused the kernal and code from labwork 4 (scale)


- Kernel: from the kernel of labwork 4:

    + Calculate the grey value as the same function in labwork 4
    + Put a threshold, if grey value is equal or larger than the threshold (180), set the value of pixels to 255, otherwise set them to 0

    --> By this way, all the pixels in the input image will turns into either while or black.

__global__ void grayscaleBinary(uchar3 *input, uchar3 *outputGrey, int width) {
        int x = threadIdx.x + blockIdx.x * blockDim.x;
        int y = threadIdx.y + blockIdx.y * blockDim.y;
        int tid = y * width + x;
        unsigned char grey = (input[tid].x + input[tid].y + input[tid].z) / 3;
        if (grey >= 180) { outputGrey[tid].x = outputGrey[tid].y = outputGrey[tid].z = 255; }
        else { outputGrey[tid].x = outputGrey[tid].y = outputGrey[tid].z = 0; }
}

2. Speedup:
vund@ictserver3:~/advancedhpc2017/labwork/build$ ./labwork 6 ../data/eiffel.jpg
USTH ICT Master 2017, Advanced Programming for HPC.
Warming up...
Starting labwork 6
labwork 6 ellapsed 96.6ms


\end{verbatim}

\end{section}


\end{section}


\end{document}
