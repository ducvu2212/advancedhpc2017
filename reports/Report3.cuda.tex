\documentclass{report}
\title{Report 3: Cuda}
\author{Nguyen Duc Vu}

\begin{document}

\maketitle

\begin{section}


\begin{verbatim}

1. Implementation for the labwork 3:

- Implement the kernel so that Host CPU can used to tell Device what to do (I get the kernel from the lecture slide)
__global__ void grayscale(uchar3 *input, uchar3 *output) {
int tid = threadIdx.x + blockIdx.x * blockDim.x;
output[tid].x = (input[tid].x + input[tid].y +
input[tid].z) / 3;
output[tid].z = output[tid].y = output[tid].x;
}

- Then, on the Labwork::labwork3_GPU() function, initial the variables used:
int pixelCount = inputImage->width * inputImage->height;
int blockSize = 64;
int numBlock = pixelCount / blockSize;
char *devInput, *devOutput;
- allocate the memory for input and output image on Host
cudaMalloc(&devInput, pixelCount * 3 * sizeof(char));
cudaMalloc(&devOutput, pixelCount * 3 * sizeof(char));

- Copy the image from Host to Device's memory:
cudaMemcpy(devInput, inputImage->buffer, pixelCount * 3 * sizeof(char), cudaMemcpyHostToDevice);

- Send the instruction of what need to do on Device:
grayscale<<<numBlock, blockSize>>>(devInput, devOutput);

- Free up the memory:
cudaFree(devInput);
cudaFree(devOutput);

2. Speed up:
- Labwork1: 
USTH ICT Master 2017, Advanced Programming for HPC.
Warming up...
Starting labwork 1
labwork 1 CPU ellapsed 3042.8ms
labwork 1 ellapsed 794.6ms


vund@ictserver3:~/advancedhpc2017/labwork/build$ ./labwork 3 ../data/eiffel.jpg
USTH ICT Master 2017, Advanced Programming for HPC.
Warming up...
Starting labwork 3
labwork 3 ellapsed 130.1ms

--> The function labwork3 is must faster then labwork1 (around 7 times faster in compared with labwork1)

\end{verbatim}

\end{section}


\end{section}


\end{document}
